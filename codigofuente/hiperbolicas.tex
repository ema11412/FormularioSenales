\chapter*{Funciones hiperbólicas}



\begin{minipage}[t]{0.5\textwidth}

\section*{Definición de las funciones}

\begin{align*}
& \senh x = \frac{e^x-e^{-x}}{2}\\
& \cosh x = \frac{e^x+e^{-x}}{2}\\
& \tanh x = \frac{e^x-e^{-x}}{e^x+e^{-x}}\\
& \coth x = \frac{e^x+e^{-x}}{e^x-e^{-x}}\\
& \sech x = \frac{2}{e^x+e^{-x}}\\
& \csch x = \frac{2}{e^x-e^{-x}}
\end{align*}

\end{minipage}  \begin{minipage}[t]{0.5\textwidth}
\section*{Relaciones entre funciones}
\begin{align*}
& \tanh x = \frac{\senh x}{\cosh x} \\[5pt]
& \coth x = \frac{\cosh x}{\senh x} \\[5pt]
& \sech x = \frac{1}{\cosh x}  \\[5pt]
& \csch x = \frac{1}{\senh x} \\[5pt]
& \cosh^2 x - \senh^2 x = 1 \\[5pt]
& \sech ^2 x + \tanh^2 x = 1 \\[5pt]
& \coth ^2 x - \csch ^2 x = 1 
\end{align*}
\end{minipage} 

\section*{Funciones de ángulos negativos}

\begin{minipage}[c]{0.3\textwidth}
$$ \senh (-x)=-\senh x$$

$$ \csch (-x)=-\csch x$$
\end{minipage}  \begin{minipage}[c]{0.3\textwidth}
$$\cosh (-x)=\cosh x$$

$$\sech (-x)=\sech x$$

\end{minipage}  \begin{minipage}[c]{0.3\textwidth}
$$\tanh (-x)=-\tanh x$$

$$\coth (-x)=-\coth x$$

\end{minipage}

\section*{Formulas de adicion}
\begin{minipage}[c]{0.5\textwidth}
\begin{align*}
\senh (x \pm y) &= \senh x \cosh y \pm \cosh x \senh y\\\\
\cosh (x \pm y) &= \cosh x \cosh y \pm \senh x \senh y
\end{align*}

\end{minipage} 
\begin{minipage}[c]{0.5\textwidth}
\begin{align*}
\tanh (x \pm y) &= \frac{\tanh x \pm \tanh y}{1 \pm \tanh x \tan y}\\\\
\coth (x \pm y) &= \frac{\coth x \coth y \pm 1}{\coth x \pm \coth y}
\end{align*}
\end{minipage}

\section*{Formulas del ángulo doble}
\begin{align*}
	\senh 2x &= 2\senh x \cosh x \\[5pt]
	\cosh 2x &= \cosh^2 x +\senh^2 x = 1+2\senh^2 x = 2\cosh^2 x - 1 \\
	\tanh 2x &= \frac{2\tanh x}{1+\tanh^2 x}
\end{align*}

\section*{Formulas de medio ángulo}

\begin{align*}
	\senh \frac{A}{2} &= \pm\sqrt{\frac{\cosh x -1}{2}}\\
	\cosh \frac{A}{2} &= \sqrt{\frac{\cosh x +1}{2}}\\
	\tanh \frac{A}{2} &= \pm\sqrt{\frac{\cosh x -1}{\cosh x+1}}\\
					  &= \frac{\senh x}{\cosh x +1} = \frac{\cosh x -1}{\senh x}
\end{align*}
\begin{center}
($+$ si $x>0$, $-$ si $x<0$)
\end{center}

\section*{Formulas de ángulos múltiplos}

\begin{align*}
	\senh 3x &= 3\senh x + 4\senh^3 x \\[5pt]
	\cosh 3x &= 4\cosh^3 x - 3\cosh x \\
	\tanh 3x &= \frac{3\tanh x+\tanh^3 x}{1+3\tanh^2 x}\\
	\senh 4x &= 4\senh x \cosh^3 x + 4 \senh^3 x \cosh x \\[5pt]
	\cosh 4x &= 8\cosh^4 x - 8\cosh^2 x + 1 \\
	\tanh 4x &= \frac{4\tanh x + 4\tanh^3 x}{1+6\tanh^2 x + \tanh^4 x}\\
\end{align*}

\section*{Potencias de funciones trigonométricas}

\begin{minipage}[c]{0.33\textwidth}
\begin{align*}
\senh^2 x &= \frac{1}{2} \cosh 2x -\frac{1}{2}\\\\
\cosh^2 x &= \frac{1}{2}+ \frac{1}{2} \cosh 2x
\end{align*}

\end{minipage} 
\begin{minipage}[c]{0.33\textwidth}
\begin{align*}
\senh^3 x &= \frac{1}{4} \senh 3x - \frac{3}{4}\senh x\\\\
\cosh^3 x &= \frac{3}{4}\cosh x+ \frac{1}{4} \cosh 3x
\end{align*}
\end{minipage}
\begin{minipage}[c]{0.33\textwidth}
\begin{align*}
\senh^4 x &= \frac{3}{8}- \frac{1}{2} \cosh 2x + \frac{1}{8}\cosh 4x\\\\
\cosh^4 x &= \frac{3}{8}+ \frac{1}{2} \cosh 2x + \frac{1}{8}\cosh 4x
\end{align*}
\end{minipage}

\section*{Suma, diferencia y producto de las funciones}

\begin{align*}
&\senh x + \senh y = 2\senh\left( \frac{x+y}{2} \right)\cosh\left( \frac{x-b}{2}\right)\\
&\senh x - \senh y = 2\cosh\left( \frac{x+y}{2} \right)\senh\left( \frac{x-y}{2}\right)\\
&\cosh x + \cosh y = 2\cosh\left( \frac{x+y}{2} \right)\cosh\left( \frac{x-y}{2}\right)\\
&\cosh x - \cosh y = 2\senh\left( \frac{x+y}{2} \right)\senh\left( \frac{x-y}{2}\right)\\
&\senh x \senh y = \frac{1}{2}\left[ \cosh (x+y)-\cosh (x-y) \right]\\
&\cosh x \cosh y = \frac{1}{2}\left[ \cosh (x+y)+\cosh (x-y) \right]\\
&\senh x \cosh y = \frac{1}{2}\left[ \senh (x+y)+\senh (x-y) \right]
\end{align*}

\section*{Relaciones entre funciones hiperbólicas}

\begin{table}[htb]
\extrarowheight = -0.5ex
\renewcommand{\arraystretch}{1.9}
\centering
\begin{tabular}{|c|c|c|c|c|c|c|}
\hline
Función & $\senh A = u$ & $\cosh A = u $ & $\tanh A = u$ & $\coth A = u$ & $\sech A = u$ & $\csch A = u$ \\ \hline
$\senh A$ & $u$ & $\sqrt{u^2-1}$ & $\frac{u}{\sqrt{1-u^2}}$ & $\frac{1}{\sqrt{u^2-1}}$ & $\frac{\sqrt{1-u^2}}{u}$ & $\frac{1}{u}$ \\ \hline
$\cosh A$ & $\sqrt{1+u^2}$ & $u$ & $\frac{1}{\sqrt{1-u^2}}$ & $\frac{u}{\sqrt{u^2-1}}$ & $\frac{1}{u}$ & $\frac{\sqrt{1+u^2}}{u}$\\ \hline
$\tanh A$ & $\frac{u}{\sqrt{1+u^2}}$ & $\frac{\sqrt{u^2-}}{u}$ & $u$ & $\frac{1}{u}$ & $\sqrt{1-u^2}$ & $\frac{1}{\sqrt{1+u^2}}$\\ \hline
$\coth A$ & $\frac{\sqrt{u^2+1}}{u}$ & $\frac{u}{\sqrt{u^2-1}}$ & $\frac{1}{u}$ & $u$ & $\frac{1}{\sqrt{1-u^2}}$ & $\sqrt{1+u^2}$\\ \hline
$\sech A$ & $\frac{1}{\sqrt{1+u^2}}$ & $\frac{1}{u}$ & $\sqrt{1-u^2}$ & $\frac{\sqrt{u^2-1}}{u}$ & $u$ & $\frac{u}{\sqrt{1+u^2}}$\\ \hline
$\csch A$ & $\frac{1}{u}$ & $\frac{1}{\sqrt{u^2-1}}$ & $\frac{\sqrt{1-u^2}}{u}$ & $\sqrt{u^2-1}$ & $\frac{u}{\sqrt{1-u^2}}$ & $u$\\ \hline
\end{tabular}
\end{table}

\newpage

\section*{Funciones hiperbólicas inversas}

\begin{minipage}[c]{0.5\textwidth}
\begin{align*}
\senh^{-1} x &= \ln \left( x+ \sqrt{x^2-1} \right)\\
\tanh^{-1} x &= \frac{1}{2}\ln \left( \frac{1+x}{1-x} \right)\\
\sech^{-1} x &= \ln \left( \frac{1}{x}+ \sqrt{\frac{1}{x^2}-1} \right)
\end{align*}
\end{minipage}\begin{minipage}[c]{0.5\textwidth}
\begin{align*}
\cosh^{-1} x &= \ln \left( x+ \sqrt{x^2+1} \right)\\
\coth^{-1} x &= \frac{1}{2}\ln \left( \frac{x+1}{x-1} \right)\\
\csch^{-1} x &= \ln \left( \frac{1}{x}+ \sqrt{\frac{1}{x^2}+1} \right)
\end{align*}
\end{minipage}

\section*{Relaciones entre hiperbólicas inversas}

\begin{minipage}[t]{0.5\textwidth}
\begin{align*}
 \csch^{-1} x &= \senh^{-1} \left( \frac{1}{x} \right)\\[3pt]
 \sech^{-1} x &= \cosh^{-1} \left( \frac{1}{x} \right)\\[3pt]
 \coth^{-1} x &= \tanh^{-1} \left( \frac{1}{x} \right)\\[3pt]
\end{align*}
\end{minipage} 
\begin{minipage}[t]{0.5\textwidth}
\begin{align*}
 \senh^{-1} (-x) &= -\senh^{-1} x\\[3pt]
 \tanh^{-1} (-x) &= -\tanh^{-1} x\\[3pt]
 \sech^{-1} (-x) &= -\sech^{-1} x\\[3pt]
 \csch^{-1} (-x) &= -\csch^{-1} x
\end{align*}
\end{minipage}


\section*{Relaciones entre funciones trigonométricas con hiperbólicas}

\begin{minipage}[t]{0.3\textwidth}
\begin{align*}
& \sen (jx) = j\senh x\\[3pt]
& \csc(jx) = -j\csch x\\[3pt]
& \senh (jx) = j\sen x\\[3pt]
& \csch(jx) = -j\csc x
\end{align*}
\end{minipage} 
\begin{minipage}[t]{0.3\textwidth}
\begin{align*}
& \cos (jx) = \cosh x\\[3pt]
& \sec(jx) = \sech x\\[3pt]
& \cosh (jx) = \cos x\\[3pt]
& \sech(jx) = \sec x
\end{align*}
\end{minipage}
\begin{minipage}[t]{0.3\textwidth}
\begin{align*}
& \tan (jx) = j\tan x\\[3pt]
& \cot(jx) = -j\coth x\\[3pt]
& \tanh (jx) = j\tan x\\[3pt]
& \coth(jx) = -j\cot x
\end{align*}
\end{minipage}


\section*{Periodicidad de funciones hiperbólicas}

\begin{minipage}[t]{0.3\textwidth}
\begin{align*}
& \senh (x+2k\pi j) = \senh x\\[4pt]
& \csch (x+2k\pi j) = \csch x
\end{align*}
\end{minipage} 
\begin{minipage}[t]{0.3\textwidth}
\begin{align*}
& \cosh (x+2k\pi j) = \cosh x\\[4pt]
& \sech (x+2k\pi j) = \sech x
\end{align*}
\end{minipage}
\begin{minipage}[t]{0.3\textwidth}
\begin{align*}
& \tanh (x+k\pi j) = \tanh x\\[4pt]
& \coth (x+k\pi j) = \coth x
\end{align*}
\end{minipage}\\
\begin{center}
Con $k \in \mathbb{Z}$.
\end{center}

\newpage
\section*{Relaciones entre hiperbólicas inversas y trigonométricas inversas}

\begin{minipage}[t]{0.5\textwidth}
\begin{align*}
& \senh^{-1} (jx) = j\sen^{-1} x\\[3pt]
& \cosh^{-1} x = \pm j \cos^{-1} x\\[3pt]
& \tanh^{-1} (jx) = j \tan^{-1} x\\[3pt]
 &\coth^{-1} (jx) = -j \cot^{-1} x\\[3pt]
& \sech^{-1} x = \pm j \sec^{-1} x\\[3pt]
& \csch^{-1} (jx) = -j \csc^{-1} x
\end{align*}
\end{minipage} 
\begin{minipage}[t]{0.5\textwidth}
\begin{align*}
& \sen^{-1} (jx) = j\senh^{-1} x\\[3pt]
& \cos^{-1} x = \pm j \cosh^{-1} x\\[3pt]
& \tan^{-1} (jx) = j \tanh^{-1} x\\[3pt]
& \cot^{-1} (jx) = -j \coth^{-1} x\\[3pt]
& \sec^{-1} x = \pm j \sech^{-1} x\\[3pt]
& \csc^{-1} (jx) = -j \csch^{-1} x
\end{align*}
\end{minipage}


