\chapter*{Funciones exponenciales y logarítmicas}
\section*{Forma polar de los números complejos}

$$z=x+jy=r(\cos \theta + j \sen \theta) = re^{j\theta}$$
\begin{center}
Donde $r=\sqrt{x^2+y^2}$ y $\theta = \mathrm{Arg} \{z \} $ 
\end{center}



\section*{Operaciones con números complejos en forma polar}

$$\left( r_1e^{j\theta_1}\right)\left( r_2e^{j\theta_2}\right)=r_1r_2e^{j(\theta_1+\theta_2)}$$

$$\frac{r_1e^{j\theta_1}}{r_2e^{j\theta_2}}=\frac{r_1}{r_2}e^{j(\theta_1+\theta_2)}$$

$$\left( re^{j\theta}\right)^p=r^pe^{jp\theta}$$

$$\left( re^{j\theta}\right)^{\frac{1}{n}}=r^{\frac{1}{n}}e^{j\frac{\theta+2k\pi}{n}}$$

\section*{Logaritmo de un numero complejo}

$$\ln \left( re^{j\theta} \right) = \ln r + i(\theta + 2k\pi)$$

$$ \mathrm{con} \,\,\, k \in \mathbb{Z}$$